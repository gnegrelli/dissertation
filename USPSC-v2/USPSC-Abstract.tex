%% Abstract.tex
% ---
% Abstract
% ---
\begin{resumo}

\begin{flushleft} 
	\setlength{\absparsep}{0pt} % ajusta o espaçamento dos parágrafos do resumo		
	\SingleSpacing 
 	\imprimirautorabr~ ~\textbf{\imprimirtitulo}.	\imprimirdata.  \pageref{LastPage}p. 
	%Substitua p. por f. quando utilizar oneside em \documentclass
	%\pageref{LastPage}f.
	\imprimirtipotrabalho~-~\imprimirinstituicao, \imprimirlocal, 	\imprimirdata. 
\end{flushleft}

\OnehalfSpacing

This project focused on the development of a software for parameter estimation of non-linear dynamic models and its application on wind power plant equivalent model. To achieve this goal, a parameter estimation package was developed in Python 3 containing the estimation methods applied alongside the models used during the study. To represent wind power plants, a generic model well-known in the literature was chosen, based on its ability of representing said plants during transients. The method applied for estimation of the model parameters is composed of two optimization algorithms. At first, Mean-Variance Mapping Optimization, an heuristic approach, is used in order to reduce the search region around a feasible solution. Afterwards, a non-linear algorithm based on Trajectory Sensitivity is used to determine the local minimum, thus estimating the parameters of the model. The method validation was made using data from simulated systems. Also, a guided user interface was  developed for this application, aiding new users and improving its usability. Both the package and interface projects are hosted on the author's \href{https://github.com/gnegrelli}{GitHub page}.

\textbf{Keywords}: Parameter Estimation. Wind Power Plants. MVMO. Trajectory Sensitivity. Python.

\end{resumo}
