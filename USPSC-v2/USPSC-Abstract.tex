%% Abstract.tex
% ---
% Abstract
% ---
\begin{resumo}

\begin{flushleft} 
	\setlength{\absparsep}{0pt} % ajusta o espaçamento dos parágrafos do resumo		
	\SingleSpacing 
 	\imprimirautorabr~ ~\textbf{\imprimirtitleabstract}.	\imprimirdata.  \pageref{LastPage}p. 
	%Substitua p. por f. quando utilizar oneside em \documentclass
	%\pageref{LastPage}f.
	\imprimirtipotrabalho~-~\imprimirinstituicao, \imprimirlocal, 	\imprimirdata. 
\end{flushleft}

\OnehalfSpacing

This project proposes the development of a software for non-linear system's model identification, focusing on wind power plants. The chosen model  for wind power plants is well-known in the literature and is capable of representing the most common wind turbine type during both steady-state and transients. The method applied to identify the model is composed of two optimization algorithms. At the begining of the process, an heuristic approach based on Mean-Variance Mapping Optimization is used in order to reduce the parameter's search region around a possible solution. Afterward, a non-linear algorithm based on Trajectory Sensitivity is used to fine-tune the parameters. The method validation will be made using data from simulated systems. Also, a guided user interface will be developed for this application, aiding new users. All coding for this project will be made in Python.


\textbf{Keywords}: Model identification. Wind power plants. MVMO. Trajectory sensitivity. Python.

\end{resumo}
