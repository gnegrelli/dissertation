\chapter{Software Concept}

\label{ch: software}

In this chapter, the concept behind the software in development is presented. The entire software will be developed using Python, a powerful, simple and fast high-level programming language that has gained large space in various sectors of industry and academy. Its rise is due mainly to the enormous number of libraries and forums developed and maintained by the users. Some examples of libraries used in this project are numpy (for scientific computing), matplotlib (plotting library), Tkinter and Qt (Graphical User Interface toolkit). Python is also open-source, not requiring a paid software to code and most of its applications are free.

Although the estimation method and mathematical model that are subject of this work, the software in development will not be specific to them. Instead, it will be generic and both model and method may be imported as packages. This will allow future users to employ it on different applications concerning parameter estimation. Also, comparison between method's performance and model's precision can be easily done with this software.

In order to improve the experience of users, a Graphical User Interface (GUI) will be developed. It will provide an simple environment for all users, so they won't need to go through the code to change any settings. Instead, the settings will be done at the beginning of the process and follow a predefined order.

The start page will display some information about the software and the parameter estimation process. Next, the user will choose from a list which mathematical model will be employed. After that, a list of identification methods will be presented and the user will be able to pick up to two methods. The settings of the chosen methods will done on the following windows and, finally, the user will enter the file containing the real system data and point out which data will be used as input and output. With all set, the estimation process will start and at its end a report will display the estimated parameters and the comparison between real system and model behaviours. 

The order presented is not definitive and may change throughout the project if needed. However, all the steps discussed are core to the estimation process and cannot be discarded. Also, some other steps may be included in order to improve the software.