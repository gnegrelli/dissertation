%% Resumo.tex
% ---
% Resumo
% ---
\setlength{\absparsep}{18pt} % ajusta o espaçamento dos parágrafos do resumo		

\begin{resumo}[Resumo]

\begin{otherlanguage*}{brazil}

\begin{flushleft} 
	\setlength{\absparsep}{0pt} % ajusta o espaçamento da referência	
	\SingleSpacing 
	\imprimirautorabr~ ~\textbf{\imprimirtitulo}.	\imprimirdata. \pageref{LastPage}p. 
	%Substitua p. por f. quando utilizar oneside em \documentclass
	%\pageref{LastPage}f.
	\imprimirtipotrabalho~-~\imprimirinstituicao, \imprimirlocal, \imprimirdata. 
\end{flushleft}

\OnehalfSpacing 			

O presente trabalho prop\~oe o desenvovlimento de um \textit{software} voltado para a identifica\c{c}\~ao de modelos de sistemas n\~ao-lineares, com enfoque em plantas e\'olicas. O modelo escolhido para plantas e\'olicas \'e bem consolidado na literatura, sendo capaz de representar o comportamento de geradores mais utilizados nas instala\c{c}\~oes deste tipo tanto durante o regime permanente quanto em transit\'orios. O m\'etodo utilizado para a identifica\c{c}\~ao do modelo \'e constitu\'ido por dois algoritmos de otimiza\c{c}\~ao. Primeiramente, \'e empregada uma abordagem heur\'istica, baseada em Otimiza\c{c}\~ao por Mapeamento de M\'edia-Vari\^ancia, a fim de reduzir a regi\~ao de busca dos par\^ametros em torno de uma poss\'ivel solu\c{c}\~ao. Em seguida, lan\c{c}a-se m\~ao de um algoritmo n\~ao-linear, baseado no M\'etodo de Sensibilidade de Trajet\'oria, para realizar os ajustes finais nos valores dos par\^ametros. A valida\c{c}\~ao do m\'etodo ser\'a feita utilizando medidas de sistemas simulados. Com o intuito de facilitar a experi\^encia do usu\'ario com o programa, ser\'a desenvolvida uma interface gr\'afica para o \textit{software}. Tanto as rotinas para identifica\c{c}\~ao de modelos quanto a interface gr\'afica ser\~ao desenvolvidas em Python.
 

\textbf{Palavras-chave}: Identifica\c{c}\~ao de modelos. Plantas e\'olicas. MVMO. Sensibilidade de trajet\'oria. Python.

\end{otherlanguage*}

\end{resumo}