%% Resumo.tex
% ---
% Resumo
% ---
\setlength{\absparsep}{18pt} % ajusta o espaçamento dos parágrafos do resumo		

\begin{resumo}[Resumo]

\begin{otherlanguage*}{brazil}

\begin{flushleft} 
	\setlength{\absparsep}{0pt} % ajusta o espaçamento da referência	
	\SingleSpacing 
	\imprimirautorabr~ ~\textbf{\imprimirtitleabstract}.	\imprimirdata. \pageref{LastPage}p. 
	%Substitua p. por f. quando utilizar oneside em \documentclass
	%\pageref{LastPage}f.
	\imprimirtipotrabalho~-~\imprimirinstituicao, \imprimirlocal, \imprimirdata. 
\end{flushleft}

\OnehalfSpacing 			

O presente trabalho prop\~oe o desenvolvimento de um \textit{software} para identifica\c{c}\~ao de um modelo equivalente de parques e\'olicos. Para este objetivo foi escolhido um modelo gen\'erico da literatura capaz de representar de forma adequada o comportamento da planta e\'olica durante transit\'orios. O m\'etodo utilizado para a identifica\c{c}\~ao do modelo \'e constitu\'ido por dois algoritmos de otimiza\c{c}\~ao. Primeiramente, \'e empregada uma abordagem heur\'istica, baseada em Otimiza\c{c}\~ao por Mapeamento de M\'edia-Vari\^ancia, a fim de reduzir a regi\~ao de busca dos par\^ametros em torno de uma poss\'ivel solu\c{c}\~ao. Em seguida, utiliza-se um algoritmo n\~ao-linear, baseado no M\'etodo de Sensibilidade de Trajet\'oria, para determinar o m\'inimo local e estimar os valores dos par\^ametros com mais precis\~ao. A valida\c{c}\~ao do m\'etodo ser\'a feita utilizando medidas de sistemas simulados. Com o intuito de facilitar a experi\^encia do usu\'ario com o programa, ser\'a desenvolvida uma interface gr\'afica amig\'avel em Python.
 

\textbf{Palavras-chave}: Plantas e\'olicas. Estima\c{c}\~ao de par\^ametros. MVMO. Sensibilidade de trajet\'oria. Python.

\end{otherlanguage*}

\end{resumo}